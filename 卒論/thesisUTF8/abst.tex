%\documentclass{AIabst} %卒論はこちら
\documentclass[master]{AIabst} %修論はこちら
% 論文概要
\input personal
\begin{document}
\makeAbstHeader
%
%
%
\section{はじめに}
このファイルでは,知能情報工学科 (先端情報工学専攻 知能情報工学分野) の卒業論文及び修士論文概要スタイルファイルの使い方について説明する.

\section{必要なファイル}
概要作成に必要なファイルは, 
\begin{center}
\begin{tabular}{ll}
{\tt personal.tex\/} & 個人データファイル\\
{\tt abst.tex\/} & 概要\\
{\tt AIabst.cls\/} & 概要クラスファイル
\end{tabular}
\end{center}
である.概要作成時には{\tt abst.tex\/}をコンパイルすればよい.

論文本体作成に必要なファイルは,
\begin{center}
\begin{tabular}{ll}
{\tt personal.tex\/} & 個人データファイル\\
{\tt main.tex\/} & 本体\\
{\tt AIcover.sty\/}  & 表紙類スタイルファイル\\
{\tt AIthesis.sty\/} & 論文本体スタイルファイル
\end{tabular}
\end{center}
である.論文本体作成時には{\tt main.tex\/}をコンパイルすればよい.

またバインダ用の表紙作成に必要なファイルは,
\begin{center}
\begin{tabular}{ll}
{\tt cover1.tex\/} & 論文表紙\\
{\tt spine.tex\/}  & 論文背表紙 \\
{\tt AIcover.cls\/}  & 表紙類クラスファイル
\end{tabular}
\end{center}
である.表紙作成時には,{\tt cover.tex\/}および{\tt spine.tex\/}をコンパイルすればよい.

基本的には以下の 3 つのファイル
\begin{center}
\begin{tabular}{l}
{\tt personal.tex\/} \\
{\tt abst.tex\/} \\
{\tt main.tex\/}
\end{tabular}
\end{center}
を加筆・修正することで,
概要と表紙を含む論文本体\footnote{論文本体をコンパイルするときに概要のPDFを読み込んでいる.もし,main.texをコンパイルして概要(abst.pdf)が正しく出力されない場合は,PDFを編集可能なソフトウェアを利用し,中表紙と論文本体の間に概要を挿入すること.}が作成できるようになっている. 
なお,表題を2行に分けたいときには,
{\tt personal.tex\/}の題目の改行位置に$\backslash\backslash$を
挿入すること.

修論の場合はクラスファイルオプション {\tt master} を指定し,
\begin{center}
\begin{verbatim}
\documentclass[master]{AIabst}
あるいは
\documentclass[master]{AIcover}
\end{verbatim}
\end{center}
などと修正したのち,コンパイルすること.

%{\tt abst.tex\/},
%{\tt cover1.tex\/},
%{\tt cover2.tex\/},
%{\tt spine.tex\/}
%における
%名前・学生番号・指導教官・論文題目
%の細かいファイルごとの修正,特に,概要と表紙における表題の改行位置の変更は,
%それぞれのファイルの 
%ファイル読み込み部分($\backslash${\tt input\{...\}\/}の部分)
%を消去して直接書き込むことで可能である.
%この場合には,ミスが無いように十分注意すること.

\section{注意する点}
本スタイルファイルで注意する点は以下の通りである.
\begin{enumerate}
\item
卒業論文における所属部門名は以下の通りである.
なお,「○○部門」の「部門」は不要である.
\begin{itemize}
\item 知能数理学部門\\
坂本,瀬部,平田,井,石坂,下薗,乃美 
の各研究室.
\item 知能情報アーキテクチャ部門\\
久代,八杉,吉田,江本,片峯 の各研究室.
\item 知能情報メディア部門\\
榎田,岡部,嶋田,乃万,國近,中村
の各研究室.
\end{itemize}

\item
基本的に通常の LaTeX と同じように利用できる.
ただし,パッケージは最低限のものしか入れていないので,
必要に応じて{\tt abst.tex\/}へ追加すること.
\item
見出しは{\tt section\/}と{\tt subsection\/}
しか使えない.
\item
{\tt baselineskip\/}は変更しないこと.
\item
参考文献を加えてもよい.
使い方は通常通りである.
例えば,
``LaTeXの参考書には\cite{rakuraku,bibunsyo}がある.''
\end{enumerate}

{\small
\baselineskip 12pt
\begin{thebibliography}{1}

\bibitem{rakuraku}
野寺隆志,
\newblock 
楽々LATEX (第2版),
\newblock 
共立出版,1994.

\bibitem{bibunsyo}
奥村晴彦,
\newblock 
LATEX2$\varepsilon$美文書作成入門 -- 論文作成からDTPまで自由自在 --,
\newblock
技術評論社,1997.
\end{thebibliography}
}
\end{document}

